\section{Preprocessing}
Data preprocessing is a critical step in any 
machine learning application. In this project, 
we applied several preprocessing techniques to 
prepare the text data for modeling:

\begin{itemize}
    \item \textbf{Text Normalization:} 
    All text was converted to lowercase to ensure 
    uniformity. Additionally, Unicode characters were 
    transformed into ASCII to handle special characters 
    and maintain consistency across the dataset.
    \item \textbf{Tokenization:} 
    Text was split into individual words or tokens.
    \item \textbf{Stop Words Removal:} 
    Common words that do not contribute to the 
    semantic meaning were removed.
    \item \textbf{Stemming/Lemmatization:} 
    Words were reduced to their base or root form.
\end{itemize}


As shown in \autoref{fig:grape_variety_distribution},
the dataset is imbalanced. The most common
grape variety is Pinot Noir, and the least common
is Grenache.

The dataset was split into a training set
and a test set with a 80\% - 20\% ratio.

\subsection{Feature Vectorization}
To transform the text data into a numerical format
 suitable for machine learning, we used the TF-IDF 
 (Term Frequency-Inverse Document Frequency) method. 
 This approach captures the importance of words in 
 the context of the entire dataset and helps in feature 
 extraction.