\section{Modeling and Hyperparame-
ter Tuning}

Three different categories of models were
considered:
\begin{itemize}
    \item \textit{Decision Trees}
    \item \textit{Random Forests}
    \item \textit{Neural Networks}
\end{itemize}


\subsection{Imbalanced Dataset}
To overcome the class imbalance in the dataset, 
the training set was oversampled using the SMOTE 
technique. The number of records was limited to 
200,000 to control computational time and resources. 
SMOTE generates synthetic samples for the minority 
classes by interpolating between existing samples, 
which helps to balance the class distribution without 
duplicating data. This approach improves the model's 
ability to generalize, especially for underrepresented 
classes, while reducing the risk of overfitting. 
In addition to SMOTE, class weights were assigned 
to further address the imbalance during training, 
ensuring that minority classes had adequate influence 
on the model’s learning process. 

\subsection{Model Selection}
To identify the optimal hyperparameter settings for 
each model, we performed a random search by exploring 
8 randomly selected hyperparameter combinations for 
each model (Decision Trees, Random Forests, and 
Neural Networks). Each combination was evaluated 
using 6-fold cross-validation on the training set. 
The hyperparameter search spaces for the first phase 
are outlined in \autoref{tab:dt_search_space}, 
\autoref{tab:rf_search_space}, and 
\autoref{tab:nn_search_space}.

Accuracy was used as the metric to evaluate
the models.

\begin{table}[H]
    \centering
    \caption{Hyperparameter Grid for Decision Tree}
    \begin{tabular}{l l}
        \toprule
        \textbf{Hyperparameter} & \textbf{Values} \\
        \midrule
        Criterion & gini, entropy, log\_loss \\
        Min Impurity Decrease & 0.0, 1e-4, 1e-8, 1e-12 \\
        Max Depth & None, 1000 \\
        \bottomrule
    \end{tabular}
    \label{tab:dt_search_space}
\end{table}

\begin{table}[H]
    \centering
    \caption{Hyperparameter Grid for Random Forest}
    \begin{tabular}{l l}
        \toprule
        \textbf{Hyperparameter} & \textbf{Values} \\
        \midrule
        Number of Estimators & 50, 100, 150 \\
        Criterion & gini, entropy, log\_loss \\
        Min Impurity Decrease & 0.0, 1e-4, 1e-8, 1e-12 \\
        Max Depth & None, 1000 \\
        \bottomrule
    \end{tabular}
    \label{tab:rf_search_space}
\end{table}

\begin{table}[H]
    \centering
    \caption{Hyperparameter Grid for Neural Network}
    \begin{tabular}{l l}
        \toprule
        \textbf{Hyperparameter} & \textbf{Values} \\
        \midrule
        Hidden Size & 16, 32, 64 \\
        Epochs & 5, 10, 15 \\
        Learning Rate & 0.005, 0.001, 0.01 \\
        \bottomrule
    \end{tabular}
    \label{tab:nn_search_space}
\end{table}
