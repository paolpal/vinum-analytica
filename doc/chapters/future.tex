\section{Future Work}

\subsection{Enhancing the Current Approach}
Although the current models offer promising results, several improvements could be made. Fine-tuning the hyperparameters of the Neural Network model is one potential avenue for enhancing its performance. Techniques such as learning rate scheduling, advanced regularization methods, and experimenting with different network architectures could lead to better generalization and reduced overfitting. Additionally, exploring more sophisticated methods for handling class imbalance, such as ADASYN or incorporating class-specific loss functions, could further improve model accuracy, particularly for underrepresented classes.

\subsection{Regression Analysis on Wine Prices}
In addition to improving the existing classification models, future work could extend the analysis to a regression task focused on predicting wine prices. Implementing regression models to predict continuous variables, such as wine prices, would involve developing and training models specifically tailored for regression tasks, such as Linear Regression, Decision Trees for Regression, Random Forest Regressors, and Neural Networks with regression output layers.

The inclusion of price prediction could offer valuable insights into the factors that influence wine pricing, beyond just classification of wine varieties. This would involve preprocessing data for regression, exploring feature engineering techniques, and evaluating model performance using appropriate regression metrics (e.g., Mean Absolute Error, Mean Squared Error). Combining these regression results with the current classification models could provide a more comprehensive understanding of wine reviews and their impact on pricing.
